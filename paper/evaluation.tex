
\begin{figure}
  \centering
  \includegraphics[width=0.40\paperwidth]{figures/lineRateModel.pdf}
  \caption{\label{fig:p4ModelTopo} Topology of the 10 Gb/s testbed 
  for real-time TCP benchmarks, using P4 to model FEC and faulty links.}
\end{figure}

\begin{figure*}[!ht]
\centering
\begin{minipage}[b]{0.31\linewidth}
\includegraphics[width=\linewidth]{figures/lossVsTput.pdf}
\caption{Iperf throughput.}
\label{fig:lossVsTput}
\end{minipage}
\hspace{.05in}
\begin{minipage}[b]{0.31\linewidth}
\includegraphics[width=\textwidth]{figures/lossVsWindow.pdf}
\caption{Iperf TCP window sizes.}
\label{fig:lossVsWindow}
\end{minipage}
\hspace{.05in}
\begin{minipage}[b]{0.31\linewidth}
  \centering
  \includegraphics[trim=6mm 6mm 0 0, width=0.95\textwidth]{figures/latency.pdf}
\caption{Latency at error rate=0. Boxes and whiskers show quartile and 1.5*quartile ranges.}
\label{fig:latency}
\end{minipage}
\end{figure*}


\section{Evaluation}
\label{sec:evaluation}
% We evaluate \OurSys using different types of traffic to measure its improvement
% to application-level behavior in the presence of lossy links.

We evaluated \OurSys with microbenchmarks and simulation.
% to answer these questions:
%\begin{itemize}

%\item What are the resource requirements for \OurSys with different levels of 
%error correction?

%\item How much does \OurSys improve application throughput across lossy links?

%\item What benefit can \OurSys have to networks at large?
%\end{itemize}



% We
% evaluate three possible deployment models for \OurSys: an FPGA external to
% the switch; a software implementation running on the switch CPU; and an ASIC  
% implementation integrated into the switch forwarding engine. 

\subsection{\OurSys Models}

\subsubsection{Line Rate P4 Model} 
To measure the effect of faulty links and
\OurSys at the application level, we implemented a P4 model of lossy links and
the FEC encoder / decoder. The model runs at line  rate alongside the layer 2
forwarding in the Barefoot Tofinos in our  testbed. It captures three
overheads that are important to applications: the bandwidth overhead that the
encoder adds by inserting parity packets and \OurSys headers; the latency
overhead that the decoder adds when recovering lost packets; and the transport
layer overhead that faulty links add by causing packet drops.

\begin{itemize}

\item The \textbf{Encoder Model} encapsulates each packet  egressing on the
faulty link with the \OurSys header and inserts blank  parity packets into the
flow. It tracks per-port block IDs and  packet indices using P4 register
arrays. To generate parity packets, the  model clones the largest packet in
each block with the Tofino's multicast engine. The parity packets are 
delayed until all the data packets in a block egress, using recirculation. 


\item The \textbf{Faulty Link Model} adds a \emph{corruption header} to each packet
egressing the faulty link. The header indicates whether or not the neighbor
switch should consider the packet lost. The model selects packets for corruption
according to a simple binomial distribution implemented with the Tofino's
random number generator.

\item The \textbf{Decoder Model} applies to all packets ingressing from a
modeled faulty link, before any other forwarding logic. For packets that are
not tagged as corrupt, the model simply removes the \OurSys and corruption headers and
allows them to continue to the standard forwarding  pipeline. The model
recirculates corrupt packets until the next block begins, at which point it
decides whether  to recover them based on the number of non-corrupt data and
parity packets it has  observed in the block.  If it counted at least K data
plus parity packets  in the block, the model "recovers" all of the corrupt
packets by removing their corruption headers and forwarding them normally. If
recovery fails, the model simply drops the corrupt packets without forwarding.

\end{itemize}

\begin{figure}
  \centering
  \includegraphics[width=0.3\paperwidth]{figures/lossVsTput.pdf}
  \caption{\label{fig:lossVsTput} Iperf throughput at different loss rates.}
\end{figure}

Figure~\ref{fig:lossVsTput} shows the benefit that FEC has on TCP throughput
at different rates of packet loss, measured with 60 second iperf trials. With
\OurSys, Iperf sustained  over 5 Gb/s with loss up to $10^{-1}$ (1 out of
every 10 packets dropped). Without \OurSys, Iperf's throughput at that loss
rate was under 25 Mb/s.

% This figure shows the impact of H. 

Figure~\ref{fig:lossVsTput} also shows the bandwidth overhead of  FEC, which
is dominated by the number of parity packets per block ($H$).  At loss rates
greater than or equal to $10^{-4}$ the bandwidth overhead of adding  parity
packets had less of an impact on TCP throughput than lost packets, for  most
configurations tested. To reduce bandwidth overhead, the FEC  can be tuned for the
loss rate of each specific link, which is reportedly stable over
time~\cite{corropt}.


% This figure shows the impact of K.
Figure~\ref{fig:lossVsLatency} shows UDP latency statistics ...





\subsubsection{Event-based simulation}
We customized a fat-tree datacenter topology in ns-3~\cite{ns3-dcn} to
model (i)~a link with loss characteristics as described by Zhuo et
al.~\cite{Zhuo:2017:UMP:3098822.3098849}; and (ii)~FEC to support
transport protocols. In this model we experimented with end-to-end
error correction rather than link-layer, to simulate a more complex
implementation without incurring the burden of implementing it fully.

We simulated a 128-node fat-tree network with 10Gbps links where two
nodes communicate over TCP to transfer a 10MB file at 2Gbps. We found
that using FEC completely eliminated retransmissions (which consisted
  of 152, 23, and 2 packets for loss rates of $10^{-3}$, $10^{-4}$,
and $10^{-5}$ respectively). But achieving end-to-end reliability
over a lossy link with little sacrifice to latency came at a steep
end-to-end overhead of 20\%, since a parity packet was added for each
5 data packets. The approach described in this paper only adds
overhead on lossy links, rather than across paths that contain a
lossy link.


\subsection{Encoder Microbenchmarks}
\iffalse
Here we evaluate the implementation directly, not using a model.
Latency and throughput graphs for experiments involving different loss rates, and the encoder working on the CPU and FPGA.
Note: we have not optimized the CPU implementation.
\fi
We measured the performance of our current encoder implementation.  Packets
in a single flow with uniformly-distributed payload sizes of 64--1450 bytes are
supplied to the FPGA with the packet generator of DPDK 17.08.1. The encoder
processes the packet with parameters $k = 50$ and $h = 1$.
At the output, we measured a throughput of 9.3 Gbps over a 10-minute period,
nearly saturating the 10-Gbps link.

\iffalse
To ensure \OurSys's effect observed in the model are practical, we directly measured the 
full throughput of our encoder implementation in FPGA. For our benchmark, the encoder is configured to use
k=8 and h=4. Packets are generated by a tool based
on DPDK library, and are fed to the board (as is specified above~(\S\ref{sec:implementation})) through
a 10Gbps link. The average outgoing throughput measured during a 10 minutes test is 9.025Gbps.
Considering the overhead from other parts of the system, we believe the link is actually close
to being saturated, which is our basic assumption in evaluations.
\fi

For comparison, we also evaluated a reference CPU implementation, not 
optimized for performance. In the same deployment, its throughput 
measured 227Mbps on a single core, and 1399Mbps using all 8 physical cores of
our Xeon E5-2450L running at 1.8 GHz.

% As a contrast, we also evaluated a CPU implementation, which was not
% optimized for performance, and intended as a reference
% implementation. Under the same deployment, the throughput measured is



\subsection{FPGA resource consumption}
Table~\ref{tab:microbenchmarks} shows the resource requirements for the FPGA implementations of
\OurSys with different $k$ and $h$ parameters.  The resource requirements are
post-implementation utilization values reported by Xilinx Vivado.  We observe
that varying $k$ has a negligible effect on resource consumption, whereas BRAM
consumption has a strong dependence on $h$.  We believe that the BRAM consumption
can be further reduced because several arrays were overpartitioned.
%CPU cycles for the software
%implementation are measured using Linux performance counters and averaged over
%X packets,
%The timing statistics are measured using ingress and egress timestamps on
%the switch.

\begin{table}
% \footnotesize
\begin{center}
\small
% \resizebox{\linewidth}{!}{
\begin{tabular}{ l r r r r } 
\toprule
$(k, h)$ & $(25, 1)$ & $(25, 5)$ & $(25,10)$ & $(50, 1)$ \\
\midrule
%\emph{Software} & & & & \\
%\cmidrule{1-1}
%Cycles & & & & \\
%Proc. Time (ns) & & & & \\
%\midrule
%\emph{FPGA} & & & & \\
%\cmidrule{1-1}
BRAM (18Kb) & 135 (7\%) & 186 (10\%) & 248 (14\%) & 135 (7\%) \\
Flip-flop & 52420 (10\%) & 53415 (10\%) & 54497 (10\%) & 52420 (10\%) \\
LUT & 31372 (11\%) & 32439 (12\%) & 33136 (12\%) & 31368 (11\%) \\
%Proc. Time (ns) & \FIXME{?} & & & \\
\bottomrule
\end{tabular}
% }
\caption{Resource requirement %Comparison
of the FPGA %and CPU
implementation of \OurSys with different configurations.  Note that BRAMs are local memories,
and LUTs (lookup tables) are programmable gates.} % typically implement logic.}
\label{tab:microbenchmarks}
\end{center}
\end{table}

%We run \OurSys in 3 configurations: outside the switch, on the switch, and in the switch.
%Time how quickly \OurSys reacts to failing links.
